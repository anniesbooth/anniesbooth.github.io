\documentclass[11pt]{article}
\usepackage{hyperref}
\usepackage{amsmath, amsfonts, amssymb, mathrsfs}
\usepackage{dcolumn}
\usepackage{caption}
\usepackage{subcaption}
\usepackage{filemod}
\usepackage{natbib}
\usepackage[ruled, vlined]{algorithm2e}
\usepackage{floatrow}
\usepackage{setspace}
\usepackage{verbatim}
\usepackage{graphicx}

\hypersetup{
    colorlinks=true,
    linkcolor=blue, 
    urlcolor=black,
    citecolor=blue, 
    }

\oddsidemargin=0.25in
\evensidemargin=0.25in
\textwidth=7in
\textheight=8.75in
\topmargin=-.5in
\addtolength{\oddsidemargin}{-.5in}
\addtolength{\evensidemargin}{-.5in}
\footskip=0.5in

\title{\vspace{-1cm} Title}
\author{Author 1 \thanks{Corresponding author: Department of Statistics, 
	Virginia Tech, {\tt email@email.edi}} 
	\and Author 2 \thanks{Department of Statistics, NC State University} 
	\and Author 3 \footnotemark[2]}
\date{\today}

\begin{document}

\maketitle

\begin{abstract} 
Insert abstract.
\end{abstract}

\noindent \textbf{Keywords:} insert keywords

%%%%%%%%%%%%%%%%%%%%%%%%%%%%%%%%%%%%%%%%%%%%%%%%%%%%%%%%%%%%%%%%%%%%%
\section{Introduction}\label{sec:intro}
%%%%%%%%%%%%%%%%%%%%%%%%%%%%%%%%%%%%%%%%%%%%%%%%%%%%%%%%%%%%%%%%%%%%%

This is an example of a parenthetical reference \citep{sauer2023active}.
\citet{sauer2023active} is an example of an in-text reference.

\bibliographystyle{jasa}
\bibliography{main}

\end{document}
